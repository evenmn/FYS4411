\documentclass[norsk,a4paper,12pt]{article}
\usepackage[utf8]{inputenc}
\usepackage{graphicx} %for å inkludere grafikk
\usepackage{verbatim} %for å inkludere filer med tegn LaTeX ikke liker
\usepackage{tabularx}
\usepackage{booktabs}
\usepackage{amsmath}
\usepackage{float}
\usepackage{color}
\usepackage{listings}
\usepackage{hyperref}
\usepackage{amsmath}

\lstset{language=c++}
\lstset{basicstyle=\small}
\lstset{backgroundcolor=\color{white}}
\lstset{frame=single}
\lstset{stringstyle=\ttfamily}
\lstset{keywordstyle=\color{red}\bfseries}
\lstset{commentstyle=\itshape\color{blue}}
\lstset{showspaces=false}
\lstset{showstringspaces=false}
\lstset{showtabs=false}
\lstset{breaklines}
\lstset{postbreak=\raisebox{0ex}[0ex][0ex]{\ensuremath{\color{red}\hookrightarrow\space}}}
\usepackage{titlesec}

\setcounter{secnumdepth}{4}

\titleformat{\paragraph}
{\normalfont\normalsize\bfseries}{\theparagraph}{1em}{}
\titlespacing*{\paragraph}
{0pt}{3.25ex plus 1ex minus .2ex}{1.5ex plus .2ex}


\title{FYS4411 - Computational Physics II\\\vspace{2mm} \Large{Project 1}}
\author{\large Dorthea Gjestvang\\ Even Marius Nordhagen}
\date\today
\begin{document}

\maketitle
\begin{abstract}
Write the abstract here
\par 

\end{abstract}


\begin{itemize}
\item Github repository containing programs and results are in: \url{https://github.com/evenmn/FYS4411/tree/master/Project%201}
\end{itemize}


\section{Introduction}
Introduction

\section{Theory}
We study a system of $N$ bosons trapped in a harmonic oscillator with the Hamiltonian given by 
\begin{equation}
\hat{H}=\sum_i^N\bigg(-\frac{\hbar^2}{2m}\nabla_i^2+V_{ext}(\vec{r}_i)\bigg)+\sum_{i<j}^NV_{int}(\vec{r}_i,\vec{r}_j)
\end{equation}
with $V_{ext}$ as the external potential, which is the harmonic oscillator potential,
and $V_{int}$ as the interaction term, which can be ignored when developing the benchmarks. 

The wavefunction is on the form 
\begin{equation}
\Psi_T(\vec{r}_1, \vec{r}_2, ..., \vec{r}_N, \alpha, \beta)=\prod_i^Ng(\alpha, \beta, \vec{r}_i)\prod_{i<j}f(a,r_{ij})
\end{equation}
where $r_{ij}=|\vec{r}_i-\vec{r}_j|$ and $g$ is assumed to be an exponential
\begin{equation}
g(\alpha, \beta, \vec{r}_i)=\exp[-\alpha(x_i^2+y_i^2+\beta z_i^2)]
\end{equation}
which is practical since
\begin{equation}
\prod_i^Ng(\alpha, \beta, \vec{r}_i)=\exp[-\alpha(x_1^2+y_1^2+\beta z_1^2+\cdots x_N^2+y_N^2+\beta z_N^2)].
\end{equation}
$\alpha$ is a variational parameter that we later use to find the energy minimum, and $\beta$ is a constant. The $f$ presented above is the correlation wave function, which is 
\begin{equation}
f(a,r_{ij})=
\begin{cases} 
   0 & r_{ij} \leq a \\
   \left(1-\frac{a}{r_{ij}}\right) & r_{ij} > a.
\end{cases}
\end{equation}
The first case we will take into account, is when $a=0$, and one might observe that $f=1$ then. Anyway, ...

We want to calculate the local energy as a function of $\alpha$, and then use Variational Monte Carlo (VMC) described in section \ref{VMC}. For the non-interacting case, the analytical expression is well-known and given by
\begin{equation}
E = \hbar\omega(n + 1/2)
\end{equation}
where $n$ is the total number of free dimensions, which gonna be an useful benchmark. The local energy is
\begin{equation}
E_L(\vec{r})=\frac{1}{\Psi_T(\vec{r})}\hat{H}\Psi_T(\vec{r})
\end{equation}
which gives the following results considering $a=0$:

INSERT ANALYTICAL EXPRESSIONS FROM A

For $a\neq0$ it gets rather more complicated, because we need to deal with the correction wave function as well. By defining
\begin{equation}
f(a, r_{ij})=\exp{\bigg(\sum_{i<j}u(r_{ij})\bigg)}
\end{equation}
and doing a change of variables
\begin{equation}
\frac{\partial}{\partial r_k}=\frac{\partial}{\partial r_k}\frac{\partial r_{kj}}{\partial r_{kj}}=\frac{\partial r_{kj}}{\partial r_k}\frac{\partial}{\partial r_{kj}}=\frac{(r_k-r_j)}{r_{kj}}\frac{\partial}{\partial r_{kj}}
\end{equation}
one will end up with
\begin{align}
E_L=\sum_k\Bigg(-\frac{1}{2}\bigg(&4\alpha^2\Big(x_k^2+y_k^2+\beta^2z_k^2-\frac{1}{\alpha}-\frac{\beta}{2\alpha}\Big)\notag\\
&-4\alpha\sum_{j\neq k}(x_k, y_k, \beta z_k)\frac{(\vec{r}_k-\vec{r}_j)}{r_{kj}}u'(r_{kj})\\
&+\sum_{ij\neq k}\frac{(\vec{r}_k-\vec{r}_j)(\vec{r}_k-\vec{r}_i)}{r_{ki}r_{kj}}u'(r_{ki})u'(r_{kj})\notag\\
&+\sum_{j\neq k}\Big(u''(r_{kj})+\frac{2}{r_{kj}}u'(r_{kj})\Big)\bigg)+V_{ext}(\vec{r}_k)\Bigg).\notag
\end{align}
This is not a pretty expression, but .. We could also split up the local energy expression 
\begin{equation}
E_{L,i}=-\frac{\hbar^2}{2m}\frac{\nabla_i^2\Psi_T}{\Psi_T}+V_{ext}(\vec{r}_i)=E_{k,i}+E_{p,i}
\end{equation}
and calculate the local energy with a numerical approach where the second derivative can be approximated by the three-point formula:
\begin{equation}
f''(x)\simeq\frac{f(x+h)-2f(x)+f(x-h)}{h^2}.
\end{equation}
In our case the position is a three dimensional vector, so we need to handle each dimension separately. However, in section \ref{CPU}, the CPU time for the analytical and numerical approach are compared. 


\section{Methods}
\subsection{Variational Monte Carlo}\label{VMC}

\subsection{Metropolis algorithm}
\subsubsection{Brute force}
\subsubsection{Importance sampling}



\section{Results}
\subsection{CPU-time}\label{CPU}
For the brute force Metropolis algorithm we developed both an analytical and a numerical method to calculate the local energy. In table (\ref{tab:BFmet}) we present the results from these calculations and the performance. All the measurements are done in three dimensions with $\alpha=0.5$ and $1e6$ Monte Carlo cycles. $a$ is fixed to zero.
\begin{table} [H]
\centering
\caption{Optimal parameters $\alpha$ and $\beta$ and resulting energy $E_{L2}$ and variance $\sigma_{E_{L2}}$ for different $\omega$ for $\Psi_{T2}$.}
\begin{tabularx}{\textwidth}{X|XX|XX} \hline
\label{tab:BFmet}
& \multicolumn{2}{X}{\textbf{Analytical}} & \multicolumn{2}{X}{\textbf{Numerical}} \\
\#Particles & $\langle E_L\rangle$ [..] & CPU-time [s] & $\langle E_L\rangle$ [..] & CPU-time [s]\\ \hline
1 & 1.50000 & 0.146392 & 1.49999 & 0.426018 \\
10 & 15.000 & 11.8992 & 14.9999 & 38.0378 \\
100 & & & & \\
500 & & & & \\ \hline
\end{tabularx}
\end{table}

\section{Discussion}

\section{Conclusion}

\end{document}
